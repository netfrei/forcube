\hypertarget{introduction-.page-title}{%
\section{Introduction \{:
.page-title\}}\label{introduction-.page-title}}

\hypertarget{nfdi-4-small-disciplins-nfdi4sd}{%
\subsection{\texorpdfstring{\textbf{NFDI 4 Small Disciplins
(NFDI4SD)}}{NFDI 4 Small Disciplins (NFDI4SD)}}\label{nfdi-4-small-disciplins-nfdi4sd}}

Diese Webseite dient der kollaborativen Erstellung des Antrags für die
Einrichtung eines NFDI4 small disciplines. Das aktuelle
Organisationsdiagramm mit angefragen Personen und Institutionen, deren
Zusage ev. noch ausstehen kann)

\begin{figure}
\centering
\includegraphics{/Volumes/GGbackup/Dropbox/2020NFDI4SD/Cube/site/assets/NFDI4SD.svg}
\caption{NFDI4SD}
\end{figure}

\hypertarget{objectives-.page-title}{%
\section{Objectives \{: .page-title\}}\label{objectives-.page-title}}

\hypertarget{small-disciplines-.page-title}{%
\subsection{Small Disciplines \{:
.page-title\}}\label{small-disciplines-.page-title}}

\hypertarget{warum-brauchen-kleine-fuxe4cher-eine-nfdi}{%
\subsection{Warum brauchen „kleine Fächer" eine
NFDI}\label{warum-brauchen-kleine-fuxe4cher-eine-nfdi}}

Die fortschreitende Digitalisierung und die Open-Access-Bewegung haben
zu einer Transformation des Forschungsprozesses und zur stärkeren
Verbreitung von Forschungsdaten geführt. Im Zuge dieser Entwicklung
haben sich wissenschaftliche Institutionen wie Universitäten,
Forschungseinrichtungen aber auch kleinere Gruppen und Akteure zur
nachhaltigen Verwaltung und zum interdisziplinären Austausch von
Forschungsdaten verpflichtet. Die Initiative zum Aufbau einer Nationalen
Forschungsdateninfrastruktur trägt diesen massiven strukturellen Wandel
Rechnung. Damit verbunden ist die systematische Planung, Sammlung,
Verarbeitung, Analyse, Archivierung, Publikation und der Austausch von
Daten unterschiedlichen Typs zur Wiederverwendung durch die
wissenschaftliche Gemeinschaft sowie durch eine breite, interessierte
Öffentlichkeit.

Hierbei waren und sind insbesondere Geistes-, Kultur- und
Sozialwissenschaften vor neue Herausforderungen gestellt, da
Forschungsdatenmanagement und standardisierter Datenaustausch oft
weniger selbstverständlich in ihre disziplinäre Kultur und Infrastruktur
integriert sind, als im Fall der Lebens- oder Naturwissenschaften.

Die hiermit verbundenen Anforderungen sind jedoch gerade für die
sogenannten kleinen Fächer eine besondere Herausforderung. Hierzu
gehören derzeit deutschlandweit über 150 Fächer,\footnote{Siehe
  \url{https://www.bmbf.de/de/kleine-faecher-grosse-potentiale-3261.html}
  und
  https://www.kleinefaecher.de/kartierung/kleine-faecher-von-a-z.html}
überwiegend geistes-, sozial- und kulturwissenschaftliche Disziplinen,
die institutionell und ihrem Selbstverständnis nach keine bloßen
Teildisziplinen eines größeren Fachbereichs darstellen. Die langfristige
Förderung und Sicherung des Fortbestehens dieser Fächer stellt ein
hochrangiges hochschulpolitisches Ziel für die Lehr- und
Forschungsinfrastruktur in Deutschland überhaupt dar, wie
Bundesbildungsministerin Anja Karliczek betont hat: „Auf die vielen
großen Fragen unserer Gesellschaft, nicht zuletzt was sie zusammenhält,
geben die Kleinen Fächer wertvolle Antworten. Sie schaffen bedeutsames
Wissen und tragen dazu bei, unser kulturelles Erbe zu
bewahren."\footnote{https://www.bmbf.de/de/zukunft-der-kleinen-faecher-sichern-7240.html}

Gleichwohl können kleine Fächer beim Aufbau von modernen
Forschungsdateninfrastrukturen nicht oder nicht in notwendigem Umfang
auf die Ressourcen jener Institutionen zugreifen, an welche sie i.d.R.
organisatorisch angebunden sind. Sie verfügen jedoch auch nicht über
notwendige eigene Mittel und Strukturen, um die spezifischen Bedürfnisse
ihrer wissenschaftlichen Gemeinschaft im Hinblick auf modernes
Forschungsdatenmanagement zu unterstützen, bestehende Praktiken an neue
Standards anzupassen und so kompatible, nutzerorientierte Konzepte für
Datenaustausch, -sicherung und -wiederverwendung zu implementieren.

Wie die vom Bundesbildungsministerium unterstützte „Arbeitsstelle
Kleiner Fächer" betont, gehört die Mehrheit der kleinen Fächer zwar den
„insgesamt weniger drittmittelfinanzierten" Geistes- und
Kulturwissenschaften an, sie weisen aber (anders als größere geistes-
und kulturwissenschaftliche Fächer) eine vergleichsweise
überdurchschnittliche Drittmittelquote auf -- sowohl hinsichtlich der
Anzahl der Anträge wie auch bei den bewilligten Mitteln --, was nicht
zuletzt als ein Indikator für einen entsprechend hohen Bedarf angesehen
werden kann.

Ein weiterer hier zu berücksichtigender Faktor ist die besonders
kollaborativ und transdisziplinär ausgerichtete Arbeitsweise kleiner
Fächer, wie es auch in einem vom BMBF geförderten Projekt zur „Dynamik
kleiner Fächer" betont wird. Davon sind nicht nur Kooperationen mehrerer
kleiner und/oder mittelgroßer Fächern betroffen, sondern auch die Lehre
und die Betreuung des wissenschaftlichen Nachwuchses: Angesichts der
thematischen Breite und Methodenvielfalt können Forschungsfragen und
-themen in kleinen Fächern -- auch angesichts wissenschaftlicher
Spezialisierung -- oftmals nur noch durch Beteiligung mehrerer
Fachdisziplinen charakterisiert und angemessen bearbeitet werden:
Innovationen werden typischerweise durch disziplinen- und
kulturübergreifende Arbeit im transnationalen Austausch erreicht und
benötigen daher in besonderem Maße die digitale Vernetzung, den
Austausch von Forschungsdaten und eine leistungsfähige und
nutzerfreundliche Infrastruktur, um dieses Ziel zu realisieren und für
die Zukunft auszubauen.

\hypertarget{kleine-fuxe4cher-formale-kriterien}{%
\subsubsection{Kleine Fächer: Formale
Kriterien}\label{kleine-fuxe4cher-formale-kriterien}}

\begin{quote}
Als Abgrenzung zu größeren Fachbereichen und Teildisziplinen hat die
„Arbeitsstelle Kleine Fächer" der JGU Mainz Kriterien erarbeitet, die
hier als Grundlage zur Bestimmung von Merkmalen, User-Profilen und
spezifischen Anforderungen bei der Nutzung von Forschungsdaten
herangezogen und erweitert werden sollen. Derzeit geht die Arbeitsstelle
von 157 kleinen Fächern und 2311 Professuren an 89 Standorten bundesweit
aus. Diese gehören sechs Fachkulturen an, die sich wiederum in 19
Fachgruppen gliedern: 1) Geisteswissenschaften; 2)
Gesundheitswissenschaften; 3) Ingenieurwissenschaften; 4) Kunst- und
Kunstwissenschaften; 5) Naturwissenschaften, Agrarwissenschaften und
Geographie; 6) Rechts-, Wirtschafts- und
Sozialwissenschaften.\footnote{Zur Aufgliederung nach Fachgruppen siehe
  Kartierungsbericht 2019, 1.3, S. 9.} Über die Hälfte der kleinen
Fächer gehören den Geisteswissenschaften an, mit einigem Abstand gefolgt
von den unter 5) sowie den unter 6) zusammengefassten
Fachkulturen.\footnote{Laut Kartierungsbericht der Arbeitsstelle von
  2019 waren 56\% Geisteswissenschaften, 18\% Naturwissenschaften,
  Agrarwissenschaften und Geographie, 10\% Rechts-, Wirtschafts- und
  Sozialwissenschaften, 13\% Kunst und Kunstwissenschaften, 9\%
  Ingenieurwissenschaften und 5\% Gesundheitswissenschaften.}
Entscheidend ist in diesem Kontext, dass die spezifischen Interessen
vieler der in den Fachkulturen zusammengefassten Fachgruppen bislang
nicht oder kaum gesondert angesprochen werden, was nicht zuletzt einer
relativ ausgeprägten Dynamik geschuldet ist: Neue kleine Fächer wie
Digital Humanities oder Biodiversität etablieren sich, während andere
aufgrund eines starken Wachstums den Status als kleines Fach verlieren
oder aber ganz in größeren Disziplinen aufgehen.
\end{quote}

\hypertarget{institutionelle-ebene}{%
\subsubsection{Institutionelle Ebene}\label{institutionelle-ebene}}

Nicht mehr als drei unbefristete Professuren pro Standort,
deutschlandweit sind bis zu zwei Ausnahmen möglich: Für den jeweiligen
Wissenschaftszweig gibt es an deutschen Universitäten eigene Professuren
mit spezifischen Denominationen.

Eigene Studiengänge mit qualifiziertem Abschluss: Der jeweilige
Wissenschaftszweig ist mit eigenen Studiengängen (Bachelor/ Master/
Magister/ Diplom/ Staatsexamen) an deutschen Universitäten vertreten.

Die Selektion und Ausbildung des wissenschaftlichen Nachwuchses, d.h.,
die Möglichkeit der Promotion sowie die Etablierung von
Juniorprofessuren mit der Aussicht auf Verstetigung bzw.
Tenure-Track-Verfahren.

\hypertarget{sozial-kommunikative-ebene}{%
\paragraph{Sozial-kommunikative
Ebene}\label{sozial-kommunikative-ebene}}

Selbstverständnis als eigenes Fach: Die Professoren und Professorinnen,
welche den jeweiligen Wissenschaftszweig an deutschen Universitäten
vertreten, verstehen diesen als eigenständiges Fach.

Fachgesellschaft: Der jeweilige Wissenschaftszweig verfügt über eine
nationale oder internationale Fachgesellschaft oder wird in
Ausnahmefällen von einer übergeordneten Fachgesellschaft klar als
eigenständiges Fach anerkannt. Dies ist nicht zuletzt die Voraussetzung
für den Zugang zu Fördermitteln verschiedener Institutionen, worauf
kleine Fächer besonders angewiesen sind.

Fachzeitschrift: Der jeweilige Wissenschaftszweig verfügt über eigene --
nationale oder internationale -- einschlägige Publikationsorgane.

\hypertarget{bausteine-kleine-fuxe4cher---small-disciplines}{%
\subsubsection{BAUSTEINE ``Kleine Fächer'' - small
disciplines}\label{bausteine-kleine-fuxe4cher---small-disciplines}}

!!! note Eine Animation der zeitlichen Entwicklung der Größe der kleinen
Fächer (mit Ballons für Zahl der Professuren über die Jahre wäre schön)

!!! example for maps of small discipline
\href{https://www.kleinefaecher.de/kartierung/kleine-faecher-von-a-z.html?tx_dmdb_monitoring\%5BdisciplineTaxonomy\%5D=200\&cHash=7926d252f10df8873f7beff649ce65e5}{Abfallwirtschaft}

\hypertarget{linkliste-zu-kleinen-fuxe4chern}{%
\subsubsection{Linkliste zu kleinen
Fächern}\label{linkliste-zu-kleinen-fuxe4chern}}

\begin{verbatim}
1. [Fachtagung](https://www.kleinefaecher.de/beitraege/blogbeitrag/dokumentation-zur-tagung-kleine-faecher-entwicklungen-strategien-perspektiven.html)
2. [Paletschek](https://www.kleinefaecher.de/fileadmin/user_upload/img/Abschlusstagung_2019_Der_Blick_der_Universitaetsgeschichte_auf_die_Kleinen_Faecher_Paletschek.pdf)
3. [Resümee Dreyer](https://www.kleinefaecher.de/fileadmin/user_upload/img/Abschlusstagung_2019_Strategische_Weiterentwicklung_Kleiner_Faecher_Dreyer.pdf)
\end{verbatim}

\hypertarget{research-data-.page-title}{%
\subsection{Research Data \{:
.page-title\}}\label{research-data-.page-title}}

\hypertarget{research-data}{%
\subsection{Research Data}\label{research-data}}

\hypertarget{epistemology-of-research-data}{%
\subsubsection{Epistemology of research
data}\label{epistemology-of-research-data}}

Forschungsdaten wurden in der Wissenschaftensgeschichte seit den ersten
Anfängen als eigenes wissenschaftliches Genre gehandelt und publiziert.
Sie sind damit keine Erfindung des digitalen Zeitalters. Datenkataloge
wie Ptolemy's Geography haben die Orte und Wege der antiken Welt
geordnet; die Handy Tables dienten vielfach kopiert allen Gelehrten zur
Konstellation der Gestirne. Kalender und damit verbundene religiöse
Feste orientierten sich an Forschungsdaten. Lexika normieren
Fachverständnis und Kommunikation. Medizines Wissen nutzte Kräuterbücher
und Rezeptsammlungen; Kompendien und Handbücher ordneten kritisch
begutachtetes Wissen der Disziplin. Gezeitentafeln regelten die
Navigation, den Handel und das Leben an Küsten und Reisen. Reisebücher
und Itinerare dokumentieren die gesammelten Erfahrungen von Erkundungen
und ihren Entdeckungen. Die textlichen Zeugen von Forschungsdaten mit
rezeptartigem Wissen ist nach groben Schätzungen weit umfangreicher
genutzt und kopiert als wissenschaftliche Werke mit eher
``theoretischen'' Inhalten, deren legendäre Autoren wie Galen, Euklid
oder Ptolemy bei der Frage nach den großen Werken der Antike als erstes
genannt werden. Umfangreiche Werke der Forschungsdaten wie die
Babylonischen Astronomischen Tagebücher waren für den weiteren Erfolg
der Wissenschaft mindestens genauso bedeutend - doch sie trugen als
kollaboratives Gemeinschaftswerk keinen Autorennamen. Dennoch wurde die
epistemische Prüfung von Forschungsdaten in allen Wissenschaftskulturen
mit größter Sorgfalt vorgenommen, archiviert und immer wieder kritisch
überprüft worden. Es läßt sich gut dafür argumentieren, dass die
Erfindung der wissenschaftlichen Bibliothek mit der Handhabung von
Forschungsdaten eng verbunden ist.

Für die konzeptionellen Anforderungen an eine moderne, digitale
Forschungsdateninfrastruktur ist der historische Rückblick lehrreich.

\hypertarget{discovery-and-justification}{%
\paragraph{Discovery and
justification}\label{discovery-and-justification}}

\hypertarget{beyond-fair}{%
\paragraph{Beyond FAIR}\label{beyond-fair}}

\hypertarget{paratext-and-citables}{%
\subsubsection{Paratext and Citables}\label{paratext-and-citables}}

Die Forderung nach einer umfassenden Erschließung aber auch einer
konsistenten Kontextualisierung wissenschaftlicher Daten ist ein
zentrales Anliegen. Hier kann eine Analogie zur \emph{Paratext}-Debatte
{[}@zotero-40395{]} der letzten Jahre hergestellt werden: Basierend auf
dem von Gérard Genette (1930-2018) eingeführten philologischen Begriff
werden Paratexte (unterteilt in \emph{Peri-} und \emph{Epitext}) als
jene einem (ursprünglich literarischen) Basistext beigefügten Elemente
bezeichnet, welche Rezeption und Vertrieb eines Werkes entscheidend
mitprägen und steuern, darunter Informationen zum Autor, Verlag,
Titelei, Vorwort, Dank etc.{[}@genette1997{]} Dieses Konzept, das
größere Aufmerksamkeit auf Produktionsprozesse und
Autorisierungsinstanzen lenkt, wurde in der weiteren Forschung
sukzessive um Textgattungen und Formate neuer Medien erweitert. Im
Kontext der Digitalisierung wissenschaftlicher Daten heißt es: Gerade
die Prüfung, Autorisierung und nachhaltige Präsentation von
Forschungsdaten wird entscheidenden Einfluss auf die Wahrung und
Weiterentwicklung nationaler wie internationaler Standards und
Qualitätsansprüche haben und sicherstellen.

\hypertarget{publication}{%
\subsubsection{Publication}\label{publication}}

\hypertarget{workflow-and-nfdi-cube}{%
\subsubsection{Workflow and NFDI Cube}\label{workflow-and-nfdi-cube}}

\textbackslash bibliography

\hypertarget{objectives-and-task-areas-.page-title}{%
\subsection{Objectives and Task Areas \{:
.page-title\}}\label{objectives-and-task-areas-.page-title}}

\hypertarget{summary-of-the-proposed-consortiums-main-objectives-and-task-areas}{%
\subsection{Summary of the proposed consortium's main objectives and
task
areas}\label{summary-of-the-proposed-consortiums-main-objectives-and-task-areas}}

Digital transformation is fundamentally changing the way researchers
work, particularly those working in the small disciplines. These
scholars are typically involved in highly collaborative and global
long-term projects which use innovative technology, but they often lack
the research data and publication infrastructure normally provided by
their home institutions. Collaborative projects require agile workflows
for all user groups. Data aggregation, preparation, processing, analyses
and publications are elements of modern scholarly research. The aim of
the NFDI4SD consortium is to provide scholars with research-integrating
data along with other scientific activities. By concentrating on the
small disciplines, the consortium will be able to provide researchers
with the services that they require on a daily basis. The first
objective will be to break up the traditional sequential research
organization and weave research data, including the publication of the
data, into an integrated research process, complemented by a moderated
research model that weaves research data into the ongoing research
workflow (the ``cube'').

\hypertarget{objectives}{%
\subsubsection{Objectives}\label{objectives}}

!!! Objective\_1 Both the use and production of research data will be
tightly integrated into ongoing research projects. From the beginning of
science research has been an collaborative enterprise. NFDI4SD services
will become integral part of current research via an initial
collaborative agreement between new research projects and the NFDI4SD.
These agreements will be open to all disciplines, independent of their
institutional classification as ``small disciplines''.

Currently, the main desideratum of researchers working in the small
disciplines is access to modern computational research data beyond the
support of their home institutions and third-party funding bodies. As
science is per se a collaborative undertaking, the open knowledge
exchange is fundamental to innovative and critical science. As a
supporting infrastructure, the NFDI4SD will be a novel institutional
research partner in the scientific arena.

!!! Objective\_2 The NFDI4SD will develop research data services that
respond directly to the feedback of research projects. For many years
now, the Arbeitsstelle Kleine Fächer has been recording the
institutional settings of small disciplines. It enables direct
communication between researchers, students, institutions and, in the
future, the NFDI4SD.

Modern concepts of computational philosophy of science will guide the
software architecture of computational research data flows.

!!! Objective\_3 The NFDI4SD will use and supplement best-practice open
source tools and computational libraries for services within the
European Open Science Cloud (EOSC).

The NFDI4SD will be an active member of the German NFDI consortium and
will strive to form collaborations with other suitable consortia. It
will commit itself to the strategic plans of the EOSC Roadmap and seek
membership in the newly created EOSC Association. The proven
infrastructure of CERN's Zenodo service will provide OpenAIRE data
publication, data harvesting and the implementation of FAIR data
principles to implement the eight ambitions of Open Science.

!!! Objective\_4 The signing of agreements with relevant stakeholders --
libraries, archives and other content providers -- on standards,
application programming interfaces (APIs) and open access via computer
networks.

The NFDI4SD governing body will seek to make operational agreements with
a number of content-providing institutions on the implementation and
accessibility of APIs for the NFDI4SD's infrastructure hub. These
agreements, using widely accepted standards, will enable researchers to
access large sets of research data from a large number of
content-holding institutions. It is expected that, within a short space
of time, the NFDI4SD will be able to provide standardized API and
interface modules. The large variety of small disciplines involved will
ensure that special collections beyond the major stakeholders will be
included in this integrated network of content providers.

!!! Objective\_5 The NFDI4SD aims to maximize the visibility of the
impact of research

New generation metrics monitors will be implemented to monitor the use
and impact of the NFDI4SD's services as part of the consortium's support
of the research projects. Daily updated monitors and impact indicators
will enable researchers to assess their collaborative global network as
well as inform the NFDI4SD's governing body of hotspots of usage and of
the need to steer users towards a particular course of action.

!!! Objective\_6 Publication and citizen science

All scientific output for general scholarly use will be regarded as
published material. Such material goes beyond putting data on a file
server: publications implement FAIR data principles and use review,
curation and scholarly assessments. The NFDI4SD intends to establish new
procedures and references for data publications in order to enhance to a
significant degree the impact of research.

\hypertarget{desiderata}{%
\subsubsection{Desiderata}\label{desiderata}}

!!! desideratum \textbf{D1} Sequential planning tends to place
publications at the end of a research process, mostly beyond funding
period.

\hypertarget{task-areas}{%
\subsubsection{Task areas}\label{task-areas}}

The nature, benefits and characteristics of research data are
surprisingly complex: data exist in many different media formats and
contribute to the information value of the subjects of research; and
researchers benefit from the rapid exchange of their publications and
research findings. A Research Management Plan will coordinate the use
and interoperability of the NFDI4SD's services. Internal and external
forms of communicating for research purposes, data storage, usage and
revision as well as full-page archiving, including the final publication
of research results and research data, will all form part of the
NFDI4SD's services.

The workflow of research activities in the context of highly
collaborative, agile scientific communities occurs at the interplay
between data and theory, using -- among other things -- computational
means. Such activities can neither be ordered as a linear sequence of
tasks, nor as a research data life cycle. The organization of the
workflow of research activities has been designed using Thomas Kuhn's
metaphor of science as a puzzle-solving activity. We describe the
pipeline of scientific processes as the transformation of input data via
research pipes into designed outcomes. This sequence of operation can be
compared to the rotations of a multidimensional research cube (solving
Rubik's cube, for example). The cube's faces represent the content,
data, skills, or means of the research activities. Each single step (aka
cube's rotation leads to a new configuration).

This puzzle-solving metaphor facilitates the orchestration of the
NFDI4SD's services into research activities by organizing and
communicating the growing repertoire of the NFDI4SD's manifold of
research data services. The cube describes the requirements, standards
and quality of data, as well as their information, metadata and
documentation, in order to assure the best usage of services. The
NFDI4SD can confidently base its operation on a wide spectrum of
standards, API norms, data formats as recommended by the European Open
Science Cloud and other standard institutions. The NFDI4SD will rapidly
develop a user-friendly service catalogue which will serve as a
graphical user interface (GUI) for researchers, who will be able to
choose their needed service, quickly apply it to their given data and
research questions and obtain their intended results.

!!! Task\_area \textbf{TA1}: research fields

The aim of this large task area is to observe at close range the
application of the NFDI4SD's services and then to recommend the
development of additional services to satisfy the demands from the
various disciplines. TA1 will be supported by a broad range of fields,
assisted by coordinators.

!!! Task\_area \textbf{TA2}: services, machine learning, big data and
data connectivity

The operational task area develops and secures the operation of the
NFDI4SD's infrastructure. Using ZENODO utilizes Europe's largest and
renowned research data publication platform and its comprehensive
computer power.

!!! Task\_area \textbf{TA3}: collaboration tools, user interfaces and
publications

Today's research environments are leading to a rising demand for cloud
collaboration services. Direct exchange is being replaced by the
publication of texts and data. User interfaces ensure that scholars and
scientists can effectively use and control data without any special
training. In addition, metadata as well as knowledge graphs, advanced
catalogues and reference tools will enable users of the NFDI4SD's
services to make the most of open scientific data.

!!! Task\_area \textbf{TA4}: standards, metadata and quality assessment

Data encoding, flow computers, computer interfaces, APIs and
publications require the widest applicable standards, norms and
metadata. Our principal investigators (PIs) are long-term members of the
key standardization committees and so will ensure the interoperability
of the research data over a long period of time.

!!! Task\_area \textbf{TA5}: institutions, governance, public outreach,
media and literacy

The highly experienced and international scientific representatives
responsible for this TA will ensure optimal information flow and the
smooth undertaking of negotiations and agreements with research
institutions worldwide. The organization of virtual conferences,
newsletters, blogs, and hopefully of physical meetings in the future,
should ensure productive and effective scientific communication and
present the research results to the general public.

!!! Task\_area \textbf{TA6}: legal aspects

Intense scientific collaborative work and the exchange of information,
including the production of publications, involve complex legal
considerations. This area will be tasked with preparing the operating
rules, information desks and consultation services. It will also
actively shape the future legal landscape of the research activities of
the digital transformation that we are currently undergoing.

\hypertarget{services-.page-title}{%
\section{Services \{: .page-title\}}\label{services-.page-title}}

\hypertarget{portfolio-.page-title}{%
\subsection{Portfolio \{: .page-title\}}\label{portfolio-.page-title}}

\hypertarget{nfdi-services}{%
\subsection{NFDI Services}\label{nfdi-services}}

\begin{figure}
\centering
\includegraphics{/Volumes/GGbackup/Dropbox/2020NFDI4SD/Cube/site/services/NFDIServices.svg}
\caption{NFDIServices}
\end{figure}

\hypertarget{governance-.page-title}{%
\section{Governance \{: .page-title\}}\label{governance-.page-title}}

\hypertarget{bodies-.page-title}{%
\subsection{bodies \{: .page-title\}}\label{bodies-.page-title}}

\hypertarget{governance-bodies}{%
\subsection{Governance bodies}\label{governance-bodies}}

\begin{figure}
\centering
\includegraphics{/Volumes/GGbackup/Dropbox/2020NFDI4SD/Cube/site/governance/bodies.svg}
\caption{Governace bodies}
\end{figure}

\hypertarget{appendix-nicht-antragsteil-.page-title}{%
\section{Appendix (nicht Antragsteil) \{:
.page-title\}}\label{appendix-nicht-antragsteil-.page-title}}

\hypertarget{bibliography-.page-title}{%
\subsection{Bibliography \{:
.page-title\}}\label{bibliography-.page-title}}

⁠\textbackslash bibliography

\hypertarget{editing-tools-.page-title}{%
\subsection{Editing tools \{:
.page-title\}}\label{editing-tools-.page-title}}

\hypertarget{nfdi-4-small-disciplins-nfdi4sd-1}{%
\subsection{NFDI 4 Small Disciplins
(NFDI4SD)}\label{nfdi-4-small-disciplins-nfdi4sd-1}}

Die Dokumente dieser Webseite dienen der kollaborativen Abfassung des
Antrags für die Einrichtung eines NFDI. Dieser Seite finden Sie Links
und weitere Hinweise auf die Bearbeitung der Dateien.

\hypertarget{editing-documents}{%
\subsubsection{Editing documents}\label{editing-documents}}

\begin{itemize}
\tightlist
\item
  Die Dokumente liegen im Unterverzeichnis docs
\end{itemize}

\begin{Shaded}
\begin{Highlighting}[]
\NormalTok{mkdocs.yml    }\CommentTok{\# The configuration file.}
\NormalTok{    docs}\OperatorTok{/}
\NormalTok{        concept}\OperatorTok{/}\NormalTok{..   }\CommentTok{\# Dateien für die Zielsetzung, Arbeitsorganisation und grundlegende Konzepte}
\NormalTok{        cube}\OperatorTok{/}\NormalTok{..      }\CommentTok{\# Infrastruktur}
\NormalTok{        usecases}\OperatorTok{/}\NormalTok{..  }\CommentTok{\# Anwendungsfällle verschiedener Fächer}
\NormalTok{        index.md     }\CommentTok{\# The documentation homepage.}
\NormalTok{        ...          }\CommentTok{\# Other markdown pages, images and other files.}
\end{Highlighting}
\end{Shaded}

\hypertarget{markdown}{%
\subsubsection{Markdown}\label{markdown}}

\begin{itemize}
\tightlist
\item
  \href{https://facelessuser.github.io/PyMdown/user-guide/markdown-syntax/}{Markdown}
\item
  https://github.com/facelessuser/pymdown-extensions/blob/master/mkdocs.yml
\item
  Attribute List:
  https://python-markdown.github.io/extensions/attr\_list/
\item
  Pages, directory:
  https://github.com/lukasgeiter/mkdocs-awesome-pages-plugin
\item
  Admonitions / Blocks / Fences/ Animations:
  https://squidfunk.github.io/mkdocs-material/reference/admonitions/
\end{itemize}

\begin{verbatim}
graph TD
    A[Hard] -->|Text| B(Round)
    B --> C{Decision}
    C -->|One| D[Result 1]
    C -->|Two| E[Result 2]
\end{verbatim}

\hypertarget{format}{%
\subsubsection{Format}\label{format}}

\begin{itemize}
\tightlist
\item
  \href{https://squidfunk.github.io/mkdocs-material/setup/changing-the-colors/}{Colors}
\item
  \href{https://squidfunk.github.io/mkdocs-material/setup/changing-the-logo-and-icons/}{Icons}
\end{itemize}

Erweiterungen wurden eingebaut, die Textverfassungen nach akademischen
Standards erlauben. Neben der Web Präsentation sind die Texte auch
direkt in ein Word Dokument und ein PDF

\begin{itemize}
\item
  \href{https://squidfunk.github.io/mkdocs-material/reference/admonitions/}{Admonitions}
\item
  \href{https://facelessuser.github.io/pymdown-extensions/extensions/smartsymbols/}{Extensions}
\item
  citations and bibliography:
\end{itemize}

{[}@piotrowski , S. 6.{]}

\hypertarget{server-befehle}{%
\subsubsection{Server Befehle}\label{server-befehle}}

\begin{itemize}
\tightlist
\item
  \texttt{mkdocs\ new\ {[}dir-name{]}} - Create a new project.
\item
  \texttt{mkdocs\ serve} - Start the live-reloading docs server.
\item
  \texttt{mkdocs\ build} - Build the documentation site.
\item
  \texttt{mkdocs\ -h} - Print help message and exit.
\end{itemize}

\hypertarget{useful-links}{%
\subsubsection{Useful links}\label{useful-links}}

\begin{itemize}
\tightlist
\item
  \href{https://github.com/squidfunk/mkdocs-material/issues/748}{mkdocs.yml}
\end{itemize}

\hypertarget{bibliography}{%
\subsection{Bibliography}\label{bibliography}}

\textbackslash bibliography

\hypertarget{textbausteine-.page-title}{%
\subsection{Textbausteine \{:
.page-title\}}\label{textbausteine-.page-title}}

\hypertarget{nfdi-kleine-fuxe4cher}{%
\section{NFDI: Kleine Fächer}\label{nfdi-kleine-fuxe4cher}}

\hypertarget{community}{%
\subsection{Community}\label{community}}

\hypertarget{institutionen-und-vertreter}{%
\paragraph{Institutionen und
Vertreter}\label{institutionen-und-vertreter}}

\begin{quote}
Text
\end{quote}

\hypertarget{corresponding-author}{%
\subparagraph{1.1.1 Corresponding author}\label{corresponding-author}}

GG

\hypertarget{date}{%
\subparagraph{1.1.2 Date}\label{date}}

\hypertarget{citation}{%
\subparagraph{1.1.3 Citation}\label{citation}}

\hypertarget{abstract}{%
\subsection{2. Abstract}\label{abstract}}

\hypertarget{warum-brauchen-kleine-fuxe4cher-eine-nfdi-arbeitstitel}{%
\paragraph{2.1 Warum brauchen „kleine Fächer" eine NFDI
(Arbeitstitel)}\label{warum-brauchen-kleine-fuxe4cher-eine-nfdi-arbeitstitel}}

Im Zuge einer durch die fortschreitende Digitalisierung Transformation
des Forschungsprozesses und einer durch die Open-Access-Bewegung
befeuerte Verbreitung und Austausch von Forschungsdaten haben sich
wissenschaftliche Institutionen wie Universitäten,
Forschungseinrichtungen aber auch kleinere Gruppen und Akteure zur
nachhaltigen Verwaltung und zum interdisziplinären Austausch von
Forschungsdaten verpflichtet. Die Initiative zum Aufbau einer Nationalen
Forschungsdateninfrastruktur trägt diesen massiven strukturellen Wandel
Rechnung. Damit verbunden ist die systematische Planung, Sammlung,
Verarbeitung, Analyse, Archivierung, Publikation und der Austausch von
Daten unterschiedlichen Typs zur Wiederverwendung durch die
wissenschaftliche Gemeinschaft sowie durch eine breite, interessierte
Öffentlichkeit.

Hierbei waren und sind insbesondere Geistes-, Kultur- und
Sozialwissenschaften vor neue Herausforderungen gestellt, da
Forschungsdatenmanagement und standardisierter Datenaustausch oft
weniger selbstverständlich in ihre disziplinäre Kultur und Infrastruktur
integriert sind, als im Fall der Lebens- oder Naturwissenschaften.

Die Forderung nach einer umfassenden Erschließung aber auch einer
konsistenten Kontextualisierung wissenschaftlicher Daten ist dabei ein
zentrales Anliegen. Hier kann eine Analogie zur
\emph{Paratext}-Debatte\footnote{Zur Problematisierung und Erläuterung
  der Positionen siehe u. a. Rockenberger, A., PhiN 76/2016, pp.~20--60,
  \url{http://web.fu-berlin.de/phin/phin76/p76t2.htm\#fz16}} der letzten
Jahre hergestellt werden: Basierend auf dem von Gérard Genette
(1930--2018) eingeführten philologischen Begriff werden Paratexte
(unterteilt in \emph{Peri-} und \emph{Epitext}) als jene einem
(ursprünglich literarischen) Basistext beigefügten Elemente bezeichnet,
welche Rezeption und Vertrieb eines Werkes entscheidend mitprägen und
steuern, darunter Informationen zum Autor, Verlag, Titelei, Vorwort,
Dank etc.\footnote{Siehe Genette, G: Paratexte: das Buch vom Beiwerk des
  Buches, Frankfurt a. M. 1989, bes. S. 9--21.} Dieses Konzept, das
größere Aufmerksamkeit auf Produktionsprozesse und
Autorisierungsinstanzen lenkt, wurde in der weiteren Forschung
sukzessive um Textgattungen und Formate neuer Medien erweitert. Im
Kontext der Digitalisierung wissenschaftlicher Daten ist ein solches
Vorgehen von zentraler Bedeutung: Gerade die Prüfung, Autorisierung und
nachhaltige Präsentation von Forschungsdaten wird entscheidenden
Einfluss auf die Wahrung und Weiterentwicklung nationaler wie
internationaler Standards und Qualitätsansprüche haben und
sicherstellen.

Die hiermit verbundenen Anforderungen sind jedoch gerade für die
sogenannten kleinen Fächer eine besondere Herausforderung. Hierzu
gehören derzeit deutschlandweit über 150 Fächer,\footnote{Siehe
  \url{https://www.bmbf.de/de/kleine-faecher-grosse-potentiale-3261.html}
  und
  https://www.kleinefaecher.de/kartierung/kleine-faecher-von-a-z.html}
überwiegend geistes-, sozial- und kulturwissenschaftliche Disziplinen,
die institutionell und ihrem Selbstverständnis nach keine bloßen
Teildisziplinen eines größeren Fachbereichs darstellen. Die langfristige
Förderung und Sicherung des Fortbestehens dieser Fächer stellt ein
hochrangiges hochschulpolitisches Ziel für die Lehr- und
Forschungsinfrastruktur in Deutschland überhaupt dar, wie
Bundesbildungsministerin Anja Karliczek betont hat: „Auf die vielen
großen Fragen unserer Gesellschaft, nicht zuletzt was sie zusammenhält,
geben die Kleinen Fächer wertvolle Antworten. Sie schaffen bedeutsames
Wissen und tragen dazu bei, unser kulturelles Erbe zu
bewahren."\footnote{https://www.bmbf.de/de/zukunft-der-kleinen-faecher-sichern-7240.html}

Gleichwohl können kleine Fächer beim Aufbau von modernen
Forschungsdateninfrastrukturen nicht oder nicht in notwendigem Umfang
auf die Ressourcen jener Institutionen zugreifen, an welche sie i.d.R.
organisatorisch angebunden sind. Sie verfügen jedoch auch nicht über
notwendige eigene Mittel und Strukturen, um die spezifischen Bedürfnisse
ihrer wissenschaftlichen Gemeinschaft im Hinblick auf modernes
Forschungsdatenmanagement zu unterstützen, bestehende Praktiken an neue
Standards anzupassen und so kompatible, nutzerorientierte Konzepte für
Datenaustausch, -sicherung und -wiederverwendung zu implementieren.

Wie die vom Bundesbildungsministerium unterstützte „Arbeitsstelle
Kleiner Fächer" betont, gehört die Mehrheit der kleinen Fächer zwar den
„insgesamt weniger drittmittelfinanzierten" Geistes- und
Kulturwissenschaften an, sie weisen aber (anders als größere geistes-
und kulturwissenschaftliche Fächer) eine vergleichsweise
überdurchschnittliche Drittmittelquote auf -- sowohl hinsichtlich der
Anzahl der Anträge wie auch bei den bewilligten Mitteln --, was nicht
zuletzt als ein Indikator für einen entsprechend hohen Bedarf angesehen
werden kann.

Ein weiterer hier zu berücksichtigender Faktor ist die besonders
kollaborativ und transdisziplinär ausgerichtete Arbeitsweise kleiner
Fächer, wie es auch in einem vom BMBF geförderten Projekt zur „Dynamik
kleiner Fächer" betont wird. Davon sind nicht nur Kooperationen mehrerer
kleiner und/oder mittelgroßer Fächern betroffen, sondern auch die Lehre
und die Betreuung des wissenschaftlichen Nachwuchses: Angesichts der
thematischen Breite und Methodenvielfalt können Forschungsfragen und
-themen in kleinen Fächern -- auch angesichts wissenschaftlicher
Spezialisierung -- oftmals nur noch durch Beteiligung mehrerer
Fachdisziplinen charakterisiert und angemessen bearbeitet werden:
Innovationen werden typischerweise durch disziplinen- und
kulturübergreifende Arbeit im transnationalen Austausch erreicht und
benötigen daher in besonderem Maße die digitale Vernetzung, den
Austausch von Forschungsdaten und eine leistungsfähige und
nutzerfreundliche Infrastruktur, um dieses Ziel zu realisieren und für
die Zukunft auszubauen.

\hypertarget{kleine-fuxe4cher-formale-kriterien-1}{%
\paragraph{2.2 Kleine Fächer: Formale
Kriterien}\label{kleine-fuxe4cher-formale-kriterien-1}}

\begin{quote}
Als Abgrenzung zu größeren Fachbereichen und Teildisziplinen hat die
„Arbeitsstelle Kleine Fächer" der JGU Mainz Kriterien erarbeitet, die
hier als Grundlage zur Bestimmung von Merkmalen, User-Profilen und
spezifischen Anforderungen bei der Nutzung von Forschungsdaten
herangezogen und erweitert werden sollen. Derzeit geht die Arbeitsstelle
von 157 kleinen Fächern und 2311 Professuren an 89 Standorten bundesweit
aus. Diese gehören sechs Fachkulturen an, die sich wiederum in 19
Fachgruppen gliedern: 1) Geisteswissenschaften; 2)
Gesundheitswissenschaften; 3) Ingenieurwissenschaften; 4) Kunst- und
Kunstwissenschaften; 5) Naturwissenschaften, Agrarwissenschaften und
Geographie; 6) Rechts-, Wirtschafts- und
Sozialwissenschaften.\footnote{Zur Aufgliederung nach Fachgruppen siehe
  Kartierungsbericht 2019, 1.3, S. 9.} Über die Hälfte der kleinen
Fächer gehören den Geisteswissenschaften an, mit einigem Abstand gefolgt
von den unter 5) sowie den unter 6) zusammengefassten
Fachkulturen.\footnote{Laut Kartierungsbericht der Arbeitsstelle von
  2019 waren 56\% Geisteswissenschaften, 18\% Naturwissenschaften,
  Agrarwissenschaften und Geographie, 10\% Rechts-, Wirtschafts- und
  Sozialwissenschaften, 13\% Kunst und Kunstwissenschaften, 9\%
  Ingenieurwissenschaften und 5\% Gesundheitswissenschaften.}
Entscheidend ist in diesem Kontext, dass die spezifischen Interessen
vieler der in den Fachkulturen zusammengefassten Fachgruppen bislang
nicht oder kaum gesondert angesprochen werden, was nicht zuletzt einer
relativ ausgeprägten Dynamik geschuldet ist: Neue kleine Fächer wie
Digital Humanities oder Biodiversität etablieren sich, während andere
aufgrund eines starken Wachstums den Status als kleines Fach verlieren
oder aber ganz in größeren Disziplinen aufgehen.
\end{quote}

\hypertarget{institutionelle-ebene-1}{%
\subparagraph{2.2.1 Institutionelle
Ebene}\label{institutionelle-ebene-1}}

Nicht mehr als 3 unbefristete Professuren pro Standort, deutschlandweit
sind bis zu 2 Ausnahmen möglich: Für den jeweiligen Wissenschaftszweig
gibt es an deutschen Universitäten eigene Professuren mit spezifischen
Denominationen.

Eigene Studiengänge mit qualifiziertem Abschluss: Der jeweilige
Wissenschaftszweig ist mit eigenen Studiengängen (Bachelor/ Master/
Magister/ Diplom/ Staatsexamen) an deutschen Universitäten vertreten.

Die Selektion und Ausbildung des wissenschaftlichen Nachwuchses, d.h.,
die Möglichkeit der Promotion sowie die Etablierung von
Juniorprofessuren mit der Aussicht auf Verstetigung bzw.
Tenure-Track-Verfahren.

\hypertarget{sozial-kommunikative-ebene-1}{%
\subparagraph{2.2.2 Sozial-kommunikative
Ebene}\label{sozial-kommunikative-ebene-1}}

Selbstverständnis als eigenes Fach: Die Professoren und Professorinnen,
welche den jeweiligen Wissenschaftszweig an deutschen Universitäten
vertreten, verstehen diesen als eigenständiges Fach.

Fachgesellschaft: Der jeweilige Wissenschaftszweig verfügt über eine
nationale oder internationale Fachgesellschaft oder wird in
Ausnahmefällen von einer übergeordneten Fachgesellschaft klar als
eigenständiges Fach anerkannt. Dies ist nicht zuletzt die Voraussetzung
für den Zugang zu Fördermitteln verschiedener Institutionen, worauf
kleine Fächer besonders angewiesen sind.

Fachzeitschrift: Der jeweilige Wissenschaftszweig verfügt über eigene --
nationale oder internationale -- einschlägige Publikationsorgane.

\hypertarget{keywords}{%
\paragraph{2.3 Keywords}\label{keywords}}

Gemeinsame Infrastruktur für national und disziplinär stark disparate
Forschungseinheiten

Starke Vernetzung: Kleine Fächer kooperieren nicht mehr nur
projektbezogen, sondern sind generell international und kollaborativ
ausgerichtet, um den thematisch-strukturell bedingten besonderen
Forschungsanforderungen gerecht zu werden (dies sollen auch die User
Cases abbilden

Nutzerorientierung (kein top-down-Prozess)

Bedürfnisanalyse und Erstellung von Nutzerprofilen für die jeweiligen
Fächer/User

Anpassung von Leistungen durch variablen Servicekatalog

Einbindung von Fachhochschulen und nicht-akademischen Gruppen /
Projekten

Auch die Lehre und Ausbildung des wiss. Nachwuchses sind mit einbezogen

Jeweils Nutzer- und Fachgruppen eingrenzen

Inhalte: Übergang von Wissenschaft und Gesellschaft durch Vermittlung
über Fachmedien

Leistungen und Services, die auf international gültigen Standards
basieren, aber individuell angepasst sind

\hypertarget{fachdisziplinen}{%
\paragraph{2.4 Fachdisziplinen}\label{fachdisziplinen}}

Die thematische und methodische Vielfalt der kleinen Fächer soll
beispielhaft anhand der User Cases vorgestellt werden. Ein besonderes
Kennzeichen Kleiner Fächer ist dabei eine transdisziplinäre Arbeitsweise
bei der Bearbeitung komplexer Forschungsfragen und -themen, welche aus
den heterogenen Fachkulturen an der Schnittstelle verschiedener
Disziplinen resultiert. Innovationen werden typischerweise disziplinen-
und kulturübergreifend erreicht und benötigen daher in besonderem Maße
eine digitale Vernetzung und spezifische Infrastruktur, um dieses Ziel
zu realisieren.

\hypertarget{fachgebiete}{%
\subparagraph{2.4.1 Fachgebiete}\label{fachgebiete}}

History and Philosophy of Science

Ur-und Frühgeschichte

Geschichte / Ethik der Medizin

Digital Humanities

Historische Sprachwissenschaft

Bibliotheks- und Informationswissenschaft

Sinologie / Japanologie

Judaistik

Lateinamerikanistik

Planetologie

((ggf. Zukunftsforschung, FU))

\hypertarget{user-cases}{%
\subparagraph{2.4.2 User Cases}\label{user-cases}}

Höhlenzeichnungen, GG

Sonnenuhren (3D-Daten), GG

Sinologie, Technik- und Wissensentwicklung im China der Frühen Neuzeit,
z.B. Seidenmanufakturen der Ming-Zeit (MPIWG, Dagmar Schäfer)

Codex Florentinus (Azteken), 16. Jh., Valery Berthoud

Logbuch Sir Francis Drake, Bayerische Staatsbibliothek/JSTOR,
Erweiterung durch kartograf. Material, Verlinkung auf Wikipedia etc.

Basler Edition der Bernoulli-Briefwechsel (18.Jh.), Hg. von Fritz Nagel
und Sulamith Gehr

Epidemiologie: Pest -- Corona, N.N. (19.Jh.)??

Briefwechsel des Physikers Theodore von Kármán, GG

Unstructured-Data-Beispiel: Email-Korrespondenz LHC CERN, Projekt Uni
Wuppertal

Exoplaneten, EU-Forschungsprojekte

\hypertarget{zeitliche-eingrenzung-der-beispiele}{%
\subparagraph{Zeitliche Eingrenzung der
Beispiele}\label{zeitliche-eingrenzung-der-beispiele}}

Vor- und Frühgeschichte

Antike

Frühe Neuzeit

18. Jahrhundert

20. Jahrhundert

21. Jahrhundert

\hypertarget{forschungsdaten-spezifika-und-beduxfcrfnisse-kleiner-fuxe4cher}{%
\subsection{2. Forschungsdaten: Spezifika und Bedürfnisse „Kleiner
Fächer"}\label{forschungsdaten-spezifika-und-beduxfcrfnisse-kleiner-fuxe4cher}}

\hypertarget{typen-von-forschungsdaten}{%
\subparagraph{2.1 Typen von
Forschungsdaten}\label{typen-von-forschungsdaten}}

Hier stehen sowohl die Datenzirkulation als auch die Nutzung durch die
User-Gruppen im Mittelpunkt

Analyse von Textquellen (Beispiel Transkribus, Digitalisate von
Handschriften), Übersetzung von Quelltexten (siehe Deepl)

Verknüpfung mit Bildquellen (z.B. Fotografien, Zeichnungen, 3D-Modelle),
Tondokumenten und audiovisuelle Quellen

Visualisierung von Zahlen und Daten (z.B. Diagrammen Tabellen)

Erweiterung durch Erfassung und Kuratierung von Metadaten

\hypertarget{anforderungen-standardisierung-und-qualituxe4tspruxfcfung-von-datenformaten}{%
\subparagraph{2.2. Anforderungen: Standardisierung und Qualitätsprüfung
von
Datenformaten}\label{anforderungen-standardisierung-und-qualituxe4tspruxfcfung-von-datenformaten}}

Etablierung von Datenformaten und Metadaten-Standards nach validierten
Qualitätsstandards

Technische Prüfung

Bereitstellung auf barrierefreien Plattformen zur Nutzung durch
Forschung und Fachmedien
